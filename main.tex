\documentclass{report}

\input{preamble.tex}

\begin{document}
\begin{titlepage}
    \thispagestyle{empty}
    \begin{center}
        \includegraphics[width = \textwidth]{pics/kpi}
        Міністерство освіти і науки України\\
        Національний технічний університет України\\
        <<Київський політехнічний інститут ім. І. Сікорського>>\\
        Інститут прикладного системного аналізу
    \end{center}
    \vspace{30mm}
    \begin{center}
        \fontsize{22}{26}\selectfont\textbf{Доповідь} \\
        з курсу <<Числові методи оптимізації>> \\
        на тему <<Симплекс-метод>>
    \end{center}
    \vspace{30mm}
    \begin{flushright}
        \textbf{Виконали} \\ 
        студенти 3 курсу \\
        Квашук Ілья \\
        Фордуй Нікіта
    \end{flushright}
    \begin{flushright}
        \textbf{Прийняла} \\
        доцент кафедри ММСА \\
        Яковлева Алла Петрівна
    \end{flushright}
    \vspace{45mm}
    \begin{center}
        \textbf{Київ 2021}
    \end{center}
\end{titlepage}

\section{Линейное программирование}
\subsection{Задача}
\begin{equation}
	f(x)=\sum_{j=1}^n c_j x_j \to \min
\end{equation}

\begin{equation}
	\begin{cases}	
		\sum_{j=1}^n a_{ij} x_j \geq b_i, &  i=1,\dots, k\\
		\sum_{j=1}^n a_{ij} x_j= b_i, & i=k+1,\dots,m
	\end{cases}
\end{equation}
\subsection{Формы}

Общая форма:

\begin{equation}
	f(x)=\sum_{j=1}^n c_j x_j \to \min
\end{equation}

\begin{equation}
\begin{cases}	
	\sum_{j=1}^n a_{ij} x_j \geq b_i, &  i=1,\dots, k\\
	\sum_{j=1}^n a_{ij} x_j= b_i, & i=k+1,\dots,m
\end{cases}
\end{equation}

Основная форма:

\begin{equation}
	f(x)=\sum_{j=1}^n c_j x_j \to \min
\end{equation}

\begin{equation}
	\begin{cases}	
		\sum_{j=1}^n a_{ij} x_j \geq b_i, i=1,\dots, m
	\end{cases}
\end{equation}

Стандартная форма:

\begin{equation}
	f(x)=\sum_{j=1}^n c_j x_j \to \min
\end{equation}

\begin{equation}
	\begin{cases}	
		\sum_{j=1}^n a_{ij} x_j \geq b_i, &  i=1,\dots, k\\
		x_j\geq0, & j=1,\dots,n
	\end{cases}
\end{equation}

Каноническая форма: 

\begin{equation}
	f(x)=\sum_{j=1}^n c_j x_j \to \min
\end{equation}
\begin{equation}
	\begin{cases}	
		\sum_{j=1}^n a_{ij} x_j= b_i, &i=1,\dots,m\\
		x_j\geq0, & j=1,\dots,n\\
		b_i\geq0, & i=1,\dots,m
	\end{cases}
\end{equation}

Переходим от общей формы к канонической: 

\begin{equation}
	-2x_1+4x_2+3x_3 \to \max
\end{equation}

\begin{equation}
	\begin{cases}	
		x_1+2x_2+x_3\leq 1\\
		2x_1-5x_2+4x_3\geq 7\\
		x_1\geq0\\
		x_2\geq0
	\end{cases}
\end{equation}
Умножим первое уравнение на (-1):
\[2x_1-4x_2+3x_3 \to \min\]
Добавим к первому условию вспомогательну неотрицательную переменную $u_1$:
\[x_1+2x_2+x_3+u_1= 1\]
Вычтем из второго условия вспомогательную неотрицательную переменную $u_2$:
\[2x_1-5x_2+4x_3-u_2= 7\]
Представим переменную $x_3$ как $x_3^1-x_3^2$, $x_3^1\geq 0 \;x_3^2\geq 0$

\[2x_1-4x_2+3x_3^1-3x_3^2 \to \min\]
\begin{equation}
	\begin{cases}	
		x_1+2x_2+x_3^1-x_3^2+u_1= 1\\
		2x_1-5x_2+4x_3^1-4x_3^2-u_2= 7\\
		x_1,x_2,x_3^1,x_3^2,u_1,u_2\geq0
	\end{cases}
\end{equation}

\end{document}
